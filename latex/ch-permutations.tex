\chapter{Permutations}

\section{Permutation Basics}

\begin{definition}[Permutation]
    The bijection -- \textit{permutation} -- of
    \begin{equation}
        \begin{matrix}
            1 &2 &3 &\cdots &n \\
            \downarrow &\downarrow &\downarrow &\cdots &\downarrow \\
            \Permutate{1} &\Permutate{2} &\Permutate{3} &\cdots &\Permutate{n}
        \end{matrix}
    \end{equation}
    
    Is denoted as
    \begin{equation}
        \begin{pmatrix}
            1 &2 &3 &\cdots &n \\
            \Permutate{1} &\Permutate{2} &\Permutate{3} &\cdots &\Permutate{n}
        \end{pmatrix}
    \end{equation}
    
    Where $\sigma \colon \set{1, \dots, n} \to \set{1, \dots, n}$ is the \textit{permutation} bijection.
\end{definition}

\begin{definition}[Counting Permutations]
    \begin{equation}
        \VSBars{S_n} \coloneqq n!
    \end{equation}
    Which is the number of different ways to permutate $n$ elements $\set{1, 2, \dots, n} \subset \Int$. Together, the different permutations for $n$ distinct elements is the \textit{symmetric group} $S_n$.
\end{definition}

\begin{remark}
    For example, with $S_3 = \set{1, 2, 3}$, there are $3! = 6$ different ways to arrange the three distinct elements
    \begin{equation}
        \begin{matrix}
            \begin{pmatrix}
                1 & 2 & 3 \\
                1 & 2 & 3
            \end{pmatrix} \quad
            \begin{pmatrix}
                1 & 2 & 3 \\
                1 & 3 & 2
            \end{pmatrix} \quad
            \begin{pmatrix}
                1 & 2 & 3 \\
                2 & 1 & 3
            \end{pmatrix}    \\[1em]
            \begin{pmatrix}
                1 & 2 & 3 \\
                2 & 3 & 1
            \end{pmatrix} \quad
            \begin{pmatrix}
                1 & 2 & 3 \\
                3 & 1 & 2
            \end{pmatrix} \quad
            \begin{pmatrix}
                1 & 2 & 3 \\
                3 & 2 & 1
            \end{pmatrix}\\
        \end{matrix}
    \end{equation}
\end{remark}

\begin{definition}[Order of Permutation]
    The \textit{order} of a permutation $\sigma$ is the smallest $k \in \PosInt$ such that
    \begin{equation}
        \sigma^{k} = \epsilon
    \end{equation}
    Where $\epsilon$ is the \textit{identity permutation}
    \begin{equation}
        \epsilon(x) = x
    \end{equation}
\end{definition}

\begin{definition}[Sign of Permutation]
    The \textit{sign} of a permutation $\Sign{\sigma} \colon \sigma \to \set{-1, +1}$ where $\sigma \in S_n$ is defined as
    \begin{equation}
        \Sign(\sigma) = (-1)^{k}
    \end{equation}
    Where $k$ is the number of \textit{disorders} within $\sigma$, the number of pairs $(x, y)$ such that $x > y \to \sigma(x) < \sigma(y)$ or the converse $x < y \to \sigma(x) > \sigma(y)$. Additionally,
    \begin{equation}
        \Sign(\sigma) = \begin{cases}
            +1 &\text{if k is even} \\
            -1 &\text{if k is odd}
        \end{cases}
    \end{equation}
\end{definition}

\begin{remark}
    For example, in
    \begin{equation*}
        \begin{pmatrix}
            1 & 2 & 3 \\
            2 & 1 & 3
        \end{pmatrix}
    \end{equation*}
    $1 < 2$ but $\sigma(1) = 2 > \sigma(2) = 1$, hence a disorder.
    
    For each $i \in \set{1, \dots, n}$, starting from $i = 1$, compare $\sigma(i)$ with $\sigma(i + 1), \dots, \sigma(n)$ and add the number of disordered pairs, then move on to $i + 1$ and compare $\sigma(i + 1)$ with $\sigma(i + 2), \dots, \sigma(n)$ and so on.
\end{remark}

\begin{theorem}[Composition of Permutation]
	\begin{equation}
		\Sign(\sigma_1 \sigma_2) \coloneqq \Sign(\sigma_1) \cdot \Sign(\sigma_2)
	\end{equation}
\end{theorem}

Where
\begin{table}[H]
\centering
\begin{tabular}{c | c c}
\toprule
$\circ$ & even & odd  \\
\midrule
even    & even & odd  \\
odd     & odd  & even \\
\bottomrule
\end{tabular}
\caption{Sign Changes on Composition}
\end{table}
