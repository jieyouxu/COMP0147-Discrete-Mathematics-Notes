\chapter{Functions}

\section{Function Basics}

\begin{definition}[Function]
    A function $f$ is a mapping from $X$ to $Y$
    \begin{equation}
        f \colon X \mapsto Y
    \end{equation}
    \begin{itemize}
        \item $\Domain{f} = X$
        \item $\Image{f} = f(X)$
    \end{itemize}
\end{definition}

\begin{definition}[Total Function]
    A function is \textit{total} if
    \begin{equation}
        \Domain{f} = X
    \end{equation}
\end{definition}

\begin{definition}[Partial Function]
    A function is \textit{partial} if
    \begin{equation}
        \Domain{f} \subseteq X
    \end{equation}
\end{definition}

\begin{definition}[Surjection]
    A function $f \colon X \mapsto Y$ is \textit{surjective} iff
    \begin{equation}
        f(X) = Y \Leftrightarrow \Forall y \in Y \colon \Exists x \in X \colon f(x) = y
    \end{equation}
    Namely each $y \in Y$ has a corresponding $x \in X$.
\end{definition}

\begin{definition}[Injection (Encodings, One-to-one)]
    A function $f \colon X \mapsto Y$ is \textit{injective} iff
    \begin{align}
        &\Forall x_1, x_2 \in X \colon x_1 \neq x_2 \to f(x_1) \neq f(x_2) \\
        \Leftrightarrow &\Forall x_1, x_2 \in X \colon f(x_1) = f(x_2) \to x_1 = x_2
    \end{align}
    Namely each distinct element $x \in X$ maps to a different element in $Y$.
\end{definition}

\begin{definition}[Bijection]
    A function $f \colon X \mapsto Y$ is \textit{bijective} iff $f$ is both \textit{injective} and \textit{surjective}.
    \begin{equation}
        \mathrm{Bijective}(f) \coloneqq \mathrm{Injective}(f) \land \mathrm{Surjective}(f)
    \end{equation}
    The \textit{inverse bijection} $\Inverse{f} \colon Y \mapsto X$ does exist.
\end{definition}

\section{Composition of Injections}
\begin{proposition}[Composition of Injection]
    Given \textit{injections} $f \colon X \mapsto Y$ and $g \colon Y \mapsto Z$, then their \textit{composition} $h \colon X \mapsto Z$ is given by
    \begin{equation}
        h(x) \coloneqq (f \circ g)(x) \coloneqq g(f(x))
    \end{equation}
    Then $h$ is also an \textit{injective} function. Namely $h = f \circ g$ where $h$ is composed from $f$ and $g$ with $f$ applied first.
\end{proposition}

\begin{proof}
    Given any $x_1, x_2 \in X$ where $x_1 \neq x_2$, then
    \begin{equation}
        f(x_1) \ne f(x_2)
    \end{equation}
    as $f$ is \textit{injective}, and thus
    \begin{equation}
        h(x_1) = g(f(x_1)) \neq g(f(x_2)) = h(x_2)
    \end{equation}
    $h$ is \textit{injective} consequently.
\end{proof}

\section{Composition of Surjection}
\begin{proposition}[Composition of Surjection]
    Given \textit{surjections} $f \colon X \mapsto Y$ and $g \colon Y \mapsto Z$, then their \textit{composition} $h \colon X \mapsto Z$ is given by
    \begin{equation}
        h(x) \coloneqq (f \circ g)(x) \coloneqq g(f(x))
    \end{equation}
    Then $h$ is also a \textit{surjective} function.
\end{proposition}

\begin{proof}
    To prove $h \colon X \mapsto Z$ is \textit{injective}, it is required to prove that
    \begin{equation}
        \Forall z \in Z \colon \Exists x \in X \colon h(x) = z
    \end{equation}
    Where $h(x) \Leftrightarrow (f \circ g)(x) \Leftrightarrow g(f(x))$.
    
    Given any element $z \in Z$ ($\Forall z \in Z$):
    \begin{enumerate}
        \item That $g \colon Y \mapsto Z$ is \textit{surjective} by definition, then $\Exists y \in Y \colon g(y) = z$.
        \item That $f \colon X \mapsto Y$ is \textit{surjective} by definition, then $\Exists x \in X \colon f(x) = y$.
    \end{enumerate}
    Then $\Forall z \in Z \colon \Exists x \in X \colon h(x) = (f \circ g)(x) = g(f(x)) = g(y) = z$ holds true.
\end{proof}

\section{Composition of Bijection}
\begin{proposition}[Composition of Bijection]
    Given \textit{bijections} $f \colon X \mapsto Y$ and $g \colon Y \mapsto Z$, then their composition $h \colon X \mapsto Z$ is given by
    \begin{equation}
        h(x) \coloneqq (f \circ g)(x) \coloneqq g(f(x))
    \end{equation}
    Then $h$ is also a \textit{bijective} function; an \textit{inverse bijection} $\Inverse{h} \colon Z \mapsto X$ also exists.
\end{proposition}

\section{Cardinality of Sets}
\begin{definition}[Cardinality]
    The number of elements in a set $X$ is denoted $\VSBars{X}$.
\end{definition}

\begin{definition}[Equal Cardinality and Bijection]
    \begin{equation}
        \VSBars{X} = \VSBars{Y}
    \end{equation}
    Holds true if there exists a \textit{bijection} $h \colon X \mapsto Y$ (one-to-one correspondence between $X$ and $Y$).
    
    Namely, $X$ and $Y$ have the same number of distinct elements, and each distinct element $x \in X$ corresponds to exactly one distinct element $y \in Y$.
\end{definition}

\begin{theorem}[Cantor-Bernstein]
    Given
    \begin{enumerate}
        \item \textit{injective} function $f \colon X \mapsto Y$
        \item \textit{injective} function $g \colon Y \mapsto X$
    \end{enumerate}
    Then there exists a \textit{bijective} function $h \colon X \mapsto Y$.
    
    Equivalently,
    \begin{equation}
        (\VSBars{X} \le \VSBars{Y}) \land (\VSBars{Y} \le \VSBars{X}) \to (\VSBars{X} = \VSBars{Y})
    \end{equation}
\end{theorem}

\begin{remark}
    Examples include countable sets, enumerable sets
    \begin{equation}
        \VSBars{\Rational} = \VSBars{\Int} = \VSBars{\NatNum} = \AlephZero
    \end{equation}
    Where the cardinality of countable sets such as the \textit{rational numbers}, \textit{integers} and the \textit{natural numbers} is denoted as "alpeh-zero" ($\AlephZero$).
    
    On the other hand, continuum such as the \textit{real numbers} are not countable and as such
    \begin{equation}
        \VSBars{\Real} > \AlephZero
    \end{equation}
\end{remark}
