\chapter{Groups}

\section{Group Basics}

A \textit{group} is an abstract collection consisting of:
\begin{itemize}
    \item A \textit{nonempty set} $G$.
    \item A \textit{binary operation} $\ast \colon G \times G \to G$.
\end{itemize}

It is defined to satisfy the following properties:
\begin{enumerate}
    \item \textbf{Closure}
    \begin{equation}
        \Forall x, y \colon x \in G \land y \in G \to x \ast y \in G
    \end{equation}
    \item \textbf{Associativity}
    \begin{equation}
        \Forall x, y, z \in G \colon (x \ast y) \ast z \equiv x \ast (y \ast z)
    \end{equation}
    \item \textbf{Neutral Element}
    \begin{equation}
        \Exists \epsilon \in G \colon \Forall x \in G \colon x \ast \epsilon \equiv \epsilon \ast x \equiv x
    \end{equation}
    \item \textbf{Invertibility}
    \begin{equation}
        \Forall x \in G \colon \Exists y \in G \colon x \ast y \equiv y \ast x \equiv \epsilon
    \end{equation}
    This $y$ which is the \textit{inverse} element of $x$ would be denoted as $\Inverse{x}$.
\end{enumerate}
