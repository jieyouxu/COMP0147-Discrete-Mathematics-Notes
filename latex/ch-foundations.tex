\section{Set Theory}

\subsection{Set Notations}
\begin{itemize}
    \item Set definition: $A = \set{a, b, c}$
    \item Set membership (element-of): $a \in A$
    \item Set builder notation: $\set{x \given x \in \Real \land x^2 = x}$
    \item Empty set: $\emptyset$
\end{itemize}

\subsection{Properties}
\begin{itemize}
    \item No structure
    \item No order
    \item No copies
\end{itemize}

For example, $a, b, c$ are references to actual objects in
\begin{equation*}
    \set{a, b, c} \Leftrightarrow \set{c, a, b} \Leftrightarrow \set{a, b, c, b}
\end{equation*}

\subsection{Set Equality}
\begin{definition}[Set Equality]
    Set $A = B$ iff:
    \begin{enumerate}
        \item $A \subseteq B$ $\implies$ $\Forall x (x \in A \to x \in B)$
        \item $B \subseteq A$ $\implies$ $\Forall y (y \in B \to y \in A)$
    \end{enumerate}
\end{definition}

\begin{remark}
    $A = B \Leftrightarrow A \subseteq B \land B \subseteq A$
\end{remark}

\subsection{Set Operations}
\begin{itemize}
    \item \textit{Union}: $A \cup B \equiv \set{x \given x \in A \lor x \in B}$
    \item \textit{Intersection}: $A \cap B \equiv \set{x \given x \in A \land x \in B}$
    \item \textit{Relative Complement}: $A \Diff B \equiv \set{x \given x \in A \land x \not\in B}$
    \item \textit{Absolute Complement}: $\AbsComplement{A} \equiv U \Diff A \equiv \set{x \given x \in U \land x \not\in A}$
    \item \textit{Symmetric Difference}: $A \Delta B \equiv (A \Diff B) \cup (B \Diff A) \equiv (A \cup B)\Diff(A \cap B)$
    \item \textit{Cartesian Product}: $A \times B \equiv \set{(x, y) \given x \in A \land y \in B}$
\end{itemize}

\subsection{Boolean Algebra}
\begin{definition}[De Morgan's Laws]
\begin{align}
    \neg (p \lor q ) &\equiv \neg p \land \neg q \\
    \neg (p \land q) &\equiv \neg p \lor \neg q 
\end{align}
\end{definition}

\begin{definition}[Idempotent Laws]
\begin{align}
    p \lor p &\equiv p \\
    p \land p &\equiv p
\end{align}

\end{definition}

\begin{definition}[Commutative Laws]
\begin{align}
    p \lor q &\equiv q \lor p \\
    p \land q &\equiv q \land p
\end{align}
\end{definition}

\begin{definition}[Associative Laws]
\begin{align}
    p \lor (q \lor r) &\equiv (p \lor q) \lor r \\
    p \land (q \land r) &\equiv (p \land q) \land r
\end{align}
\end{definition}

\begin{definition}[Distributive Laws]
\begin{align}
    p \land (q \lor r) &\equiv (p \land q) \lor (p \land r) \\
    p \lor (q \land r) &\equiv (p \lor q) \land (p \lor r)
\end{align}
\end{definition}

\begin{definition}[Identity Laws]
\begin{align}
    p \lor \False &\equiv p \\
    p \lor \True &\equiv \True \\
    p \land \True &\equiv p \\
    p \land \False &\equiv \False
\end{align}
\end{definition}

\begin{definition}[Absorption Laws]
\begin{align}
    p \lor (p \land q) &\equiv p \\
    p \land (p \lor q) &\equiv p
\end{align}
\end{definition}

\begin{definition}[Implication and Negation Laws]\ \\
\begin{itemize}
    \item \textit{Identity}: $p \to q \equiv \neg p \lor q$
    \item \textit{Counter-example}: $\neg(p \to q) \equiv p \land \neg q$
    \item \textit{Equivalences}: $p \to q \to r \equiv (p \land q) \to r \equiv q\ to (p \to r)$
    \item \textit{Absorption}:
        \subitem $p \to \True \equiv T$
        \subitem $p \to \False \equiv \neg p$
        \subitem $\True \to p \equiv p$
        \subitem $\False \to p \equiv T$
    \item \textit{Contrapositive}: $p \to q \equiv \neg q \to \neg p$
    \item \textit{Law of Excluded Middle}: 
        \subitem $p \lor \neg p \equiv \True$
        \subitem $p \land \neg p \equiv \False$
    \item \textit{Double Negation}: $\neg \neg p \equiv p$
    \item \textit{Reduction to Absurdity}: $\neg p \to \False \equiv p$
    % TODO
\end{itemize}
\end{definition}

\subsection{Set Algebra}
\begin{definition}[De Morgan's Laws]
\begin{align}
    \AbsComplement{(A \cup B)} &\equiv \AbsComplement{A} \cap \AbsComplement{B} \\
    \AbsComplement{(A \cap B)} &\equiv \AbsComplement{A} \cup \AbsComplement{B}
\end{align}
\end{definition}

\begin{definition}[Idempotent Laws]
\begin{align}
    A \cup A &\equiv A \\
    A \cap A &\equiv A
\end{align}
\end{definition}

\begin{definition}[Commutative Laws]
\begin{align}
    A \cup B &\equiv B \cup A \\
    A \cap B &\equiv B \cap A
\end{align}
\end{definition}

\begin{definition}[Associativity Laws]
\begin{align}
    A \cup (B \cup C) &\equiv (A \cup B) \cup C \\
    A \cap (B \cap C) &\equiv (A \cap B) \cap C
\end{align}
\end{definition}

\begin{definition}[Distributive Laws]
\begin{align}
    A \cap (B \cup C) &\equiv (A \cap B) \cup (B \cap C)\\
    A \cup (B \cap C) &\equiv (A \cup B) \cap (B \cup C)
\end{align}
\end{definition}

\begin{definition}[Identity Laws]
\begin{align}
    A \cup \emptyset &\equiv A \\
    A \cap \emptyset &\equiv \emptyset \\
    A \cap U &\equiv A \\
    A \cup U &\equiv U
\end{align}
\end{definition}

\begin{definition}[Absorption Laws]
\begin{align}
    A \cup (A \cap B) &\equiv A \\
    A \cap (A \cup B) &\equiv A
\end{align}
\end{definition}

\begin{definition}[Difference Identity Laws]
\begin{align}
    C \Diff (A \cup B) &\equiv (C \Diff A) \cap (C \Diff B) \\
    C \Diff (A \cap B) &\equiv (C \Diff A) \cup (C \Diff B)
\end{align}
\end{definition}

\begin{definition}[Complement-Difference Identity Law]
\begin{equation}
    C \Diff D \equiv C \cap \AbsComplement{D}
\end{equation}
\end{definition}

\begin{definition}[Double Complement Law]
\begin{equation}
    \AbsComplement{(\AbsComplement{D})} \equiv D
\end{equation}
\end{definition}

\begin{definition}[Contraposition]
\begin{align}
    C \subseteq D &\Leftrightarrow \AbsComplement{D} \subseteq \AbsComplement{C} \\
    C = D &\Leftrightarrow \AbsComplement{C} = \AbsComplement{D}
\end{align}
\end{definition}

\begin{definition}[Arbitrary Union]\ \\
    Given sets $A_1, A_2, \dots, A_n$ where $I = \set{1, 2, \dots, n}$
    \begin{equation}
        A_1 \cup A_2 \cup \dots \cup A_n \equiv \bigcup_{i \in I} A_i
    \end{equation}
    Then
    \begin{equation}
        x \in \bigcup_{i \in I} A_i \Leftrightarrow \Exists i \in I \colon x \in A_i
    \end{equation}
\end{definition}

\begin{definition}[Arbitrary Intersection]\ \\
    Given sets $A_1, A_2, \dots, A_n$ where $I = \set{1, 2, \dots, n}$
    \begin{equation}
        A_1 \cap A_2 \cap \dots \cap A_n \equiv \bigcap_{i \in I} A_i
    \end{equation}
    Then
    \begin{equation}
        x \in \bigcap_{i \in I} A_i \Leftrightarrow \Forall i \in I \colon x \in A_i
    \end{equation}
\end{definition}

\section{Functions}
\begin{definition}[Function]
    A function $f$ is a mapping from $X$ to $Y$
    \begin{equation}
        f \colon X \mapsto Y
    \end{equation}
    \begin{itemize}
        \item $\Domain{f} = X$
        \item $\Image{f} = f(X)$
    \end{itemize}
\end{definition}

\begin{definition}[Total Function]
    A function is \textit{total} if
    \begin{equation}
        \Domain{f} = X
    \end{equation}
\end{definition}

\begin{definition}[Partial Function]
    A function is \textit{partial} if
    \begin{equation}
        \Domain{f} \subseteq X
    \end{equation}
\end{definition}

\begin{definition}[Surjection]
    A function $f \colon X \mapsto Y$ is \textit{surjective} iff
    \begin{equation}
        f(X) = Y \Leftrightarrow \Forall y \in Y \colon \Exists x \in X \colon f(x) = y
    \end{equation}
    Namely each $y \in Y$ has a corresponding $x \in X$.
\end{definition}

\begin{definition}[Injection (Encodings, One-to-one)]
    A function $f \colon X \mapsto Y$ is \textit{injective} iff
    \begin{align}
        &\Forall x_1, x_2 \in X \colon x_1 \neq x_2 \to f(x_1) \neq f(x_2) \\
        \Leftrightarrow &\Forall x_1, x_2 \in X \colon f(x_1) = f(x_2) \to x_1 = x_2
    \end{align}
    Namely each distinct element $x \in X$ maps to a different element in $Y$.
\end{definition}

\begin{definition}[Bijection]
    A function $f \colon X \mapsto Y$ is \textit{bijective} iff $f$ is both \textit{injective} and \textit{surjective}.
    \begin{equation}
        \mathrm{Bijective}(f) \equiv \mathrm{Injective}(f) \land \mathrm{Surjective}(f)
    \end{equation}
    The \textit{inverse bijection} $\Inverse{f} \colon Y \mapsto X$ does exist.
\end{definition}

\subsection{Composition of Injections}
\begin{proposition}[Composition of Injection]
    Given \textit{injections} $f \colon X \mapsto Y$ and $g \colon Y \mapsto Z$, then their \textit{composition} $h \colon X \mapsto Z$ is given by
    \begin{equation}
        h(x) = g(f(x))
    \end{equation}
    Then $h$ is also an \textit{injective} function. Namely $h = g \circ f$ where $h$ is composed from $g$ and $f$ with $f$ applied first.
\end{proposition}

\begin{proof}
    Given any $x_1, x_2 \in X$ where $x_1 \neq x_2$, then
    \begin{equation}
        f(x_1) \ne f(x_2)
    \end{equation}
    as $f$ is \textit{injective}, and thus
    \begin{equation}
        h(x_1) = g(f(x_1)) \neq g(f(x_2)) = h(x_2)
    \end{equation}
    $h$ is \textit{injective} consequently.
\end{proof}

\subsection{Composition of Surjection}
\begin{proposition}[Composition of Surjection]
    Given \textit{surjections} $f \colon X \mapsto Y$ and $g \colon Y \mapsto Z$, then their \textit{composition} $h \colon X \mapsto Z$ is given by
    \begin{equation}
        h(x) = g(f(x))
    \end{equation}
    Then $h$ is also a \textit{surjective} function.
\end{proposition}

\begin{proof}
    To prove $h \colon X \mapsto Z$ is \textit{injective}, it is required to prove that
    \begin{equation}
        \Forall z \in Z \colon \Exists x \in X \colon h(x) = z
    \end{equation}
    Where $h(x) \Leftrightarrow (g \circ f)(x) \Leftrightarrow g(f(x))$.
    
    Given any element $z \in Z$ ($\Forall z \in Z$):
    \begin{enumerate}
        \item That $g \colon Y \mapsto Z$ is \textit{surjective} by definition, then $\Exists y \in Y \colon g(y) = z$.
        \item That $f \colon X \mapsto Y$ is \textit{surjective} by definition, then $\Exists x \in X \colon f(x) = y$.
    \end{enumerate}
    Then $\Forall z \in Z \colon \Exists x \in X \colon h(x) = (g \circ f)(x) = g(f(x)) = g(y) = z$ holds true.
\end{proof}

\subsection{Composition of Bijection}
\begin{proposition}[Composition of Bijection]
    Given \textit{bijections} $f \colon X \mapsto Y$ and $g \colon Y \mapsto Z$, then their composition $h \colon X \mapsto Z$ is given by
    \begin{equation}
        h(x) = g(f(x))
    \end{equation}
    Then $h$ is also a \textit{bijective} function; an \textit{inverse bijection} $\Inverse{h} \colon Z \mapsto X$ also exists.
\end{proposition}

\subsection{Cardinality of Sets}
\begin{definition}[Cardinality]
    The number of elements in a set $X$ is denoted $\VSBars{X}$.
\end{definition}

\begin{definition}[Equal Cardinality and Bijection]
    \begin{equation}
        \VSBars{X} = \VSBars{Y}
    \end{equation}
    Holds true if there exists a \textit{bijection} $h \colon X \mapsto Y$ (one-to-one correspondence between $X$ and $Y$).
    
    Namely, $X$ and $Y$ have the same number of distinct elements, and each distinct element $x \in X$ corresponds to exactly one distinct element $y \in Y$.
\end{definition}

\begin{theorem}[Cantor-Bernstein]
    Given
    \begin{enumerate}
        \item \textit{injective} function $f \colon X \mapsto Y$
        \item \textit{injective} function $g \colon Y \mapsto X$
    \end{enumerate}
    Then there exists a \textit{bijective} function $h \colon X \mapsto Y$.
    
    Equivalently,
    \begin{equation}
        (\VSBars{X} \le \VSBars{Y}) \land (\VSBars{Y} \le \VSBars{X}) \to (\VSBars{X} = \VSBars{Y})
    \end{equation}
\end{theorem}

\begin{remark}
    Examples include countable sets, enumerable sets
    \begin{equation}
        \VSBars{\Rational} = \VSBars{\Int} = \VSBars{\NatNum} = \AlephZero
    \end{equation}
    Where the cardinality of countable sets such as the \textit{rational numbers}, \textit{integers} and the \textit{natural numbers} is denoted as "alpeh-zero" ($\AlephZero$).
    
    On the other hand, continuum such as the \textit{real numbers} are not countable and as such
    \begin{equation}
        \VSBars{\Real} > \AlephZero
    \end{equation}
\end{remark}
