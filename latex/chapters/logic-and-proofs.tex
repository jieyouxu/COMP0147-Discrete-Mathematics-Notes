\section{Propositional Logic} \label{sec:propositionanl_logic}

\subsection{Propositions}

\begin{definition} \label{def:proposition}
    A \keyword{proposition} is statement which is either \textit{true} or \textit{false} but not both.
\end{definition}

Propositions can be denoted via uppercase letters, $P, Q, R, S, \dots$.

\begin{example} \label{ex:proposition}
    Let $P = \text{\say{Computer Science is life}}$.
\end{example}

\begin{definition} \label{def:negation}
    The \keyword{negation} of a proposition $P$ can be denoted as $\lnot P$ or $\bar{P}$.
\end{definition}

\begin{table}[htb]
    \centering
    \begin{tabular}[t]{cc}
        \toprule
        $P$   & $\lnot P$     \\
        \midrule
        0     & 1             \\
        1     & 0             \\
        \bottomrule
    \end{tabular}
    
    \caption{Truth Table of $\lnot P$}
    \label{table:truth_table_negation}
\end{table}

\begin{example}
    $\text{\say{Computer Science is not life}}$ can be denoted as $\lnot P$ or $\overline{P}$.
\end{example}

\begin{definition} \label{def:and}
    The \keyword{logical AND} or \keyword{conjunction} of $P$ and $Q$ can be denoted as $P \land Q$.
\end{definition}

\begin{table}[htb]
    \centering
    \begin{tabular}[t]{ccc}
        \toprule
        $P$   & $Q$ & $P \land Q$   \\
        \midrule
        0     & 0   & 0             \\
        0     & 1   & 0             \\
        1     & 0   & 0             \\
        1     & 1   & 1             \\
        \bottomrule
    \end{tabular}
    
    \caption{Truth Table of $P \land Q$}
    \label{table:truth_table_and}
\end{table}

\begin{definition} \label{def:or}
    The \keyword{logical OR} or \keyword{disjunction} of $P$ and $Q$ can be denoted as $P \lor Q$.
\end{definition}

\begin{table}[htb]
    \centering
    \begin{tabular}[t]{ccc}
        \toprule
        $P$   & $Q$ & $P \lor Q$   \\
        \midrule
        0     & 0   & 0             \\
        0     & 1   & 1             \\
        1     & 0   & 1             \\
        1     & 1   & 1             \\
        \bottomrule
    \end{tabular}
    
    \caption{Truth Table of $P \lor Q$}
    \label{table:truth_table_or}
\end{table}

\begin{definition} \label{def:xor}
    The \keyword{logical XOR} or \keyword{exclusive or} of $P$ and $Q$ can be denoted as $P \oplus Q$.
\end{definition}

\begin{table}[htb]
    \centering
    \begin{tabular}[t]{ccc}
        \toprule
        $P$   & $Q$ & $P \oplus Q$   \\
        \midrule
        0     & 0   & 0             \\
        0     & 1   & 1             \\
        1     & 0   & 1             \\
        1     & 1   & 0             \\
        \bottomrule
    \end{tabular}
    
    \caption{Truth Table of $P \oplus Q$}
    \label{table:truth_table_xor}
\end{table}

\begin{definition} \label{def:implication}
    The \keyword{implication} between $P$ and $Q$ can be denoted as $P \to Q$.
\end{definition}
\marginpar{Note that in $P \to Q$, $P$ is known as the \keyword{premise} and $Q$ is the \keyword{conclusion}.}

\begin{table}[htb]
    \centering
    \begin{tabular}[t]{ccc}
        \toprule
        $P$   & $Q$ & $P \to Q$   \\
        \midrule
        0     & 0   & 1             \\
        0     & 1   & 1             \\
        1     & 0   & 0             \\
        1     & 1   & 1             \\
        \bottomrule
    \end{tabular}
    
    \caption{Truth Table of $P \oplus Q$}
    \label{table:truth_table_implication}
\end{table}

\begin{definition} \label{def:converse}
    Given implication $P \to Q$, its \keyword{converse} is then $Q \to P$.
\end{definition}

\begin{definition} \label{def:contrapositive}
    Given implication $P \to Q$, its \keyword{contrapositive} is then $\lnot Q \to \lnot P$.
\end{definition}
\marginpar{Note that the \keyword{contrapositive} $\lnot Q \to \lnot P$ is \textit{equivalent} to the original \keyword{implication} $P \to Q$.}

\begin{definition} \label{def:biconditional}
    The \keyword{biconditional} or \keyword{bi-implication} between $P$ and $Q$ can be denoted as $P \biconditional Q$.
\end{definition}

\begin{table}[htb]
    \centering
    \begin{tabular}[t]{ccc}
        \toprule
        $P$   & $Q$ & $P \biconditional Q$   \\
        \midrule
        0     & 0   & 1             \\
        0     & 1   & 0             \\
        1     & 0   & 0             \\
        1     & 1   & 1             \\
        \bottomrule
    \end{tabular}
    
    \caption{Truth Table of $P \biconditional Q$}
    \label{table:truth_table_biconditional}
\end{table}

\subsection{Composition of Propositions}
\begin{example}
    Given a compound proposition $(P \land Q) \to (\lnot Q \lor P)$
\end{example}

\begin{table}[htb]
    \centering
    \begin{tabular}[t]{llllll}
        \toprule
        $P$ & $Q$ & $(P \land Q)$ & $\lnot Q$ & $(\lnot Q \lor P)$ & $(P \land Q) \to (\lnot Q \lor P)$ \\
        \midrule
        
        \bottomrule
    \end{tabular}
    
    \caption{Truth Table of $P \biconditional Q$}
    \label{table:truth_table_composition}
\end{table}